\documentclass[a4paper,10pt]{article}
\usepackage[T1]{fontenc}
\usepackage[latin1]{inputenc}
\usepackage{url}
\usepackage[all]{xy}
\usepackage{amsmath}
\usepackage{listings}
\lstset{language=C}

\title{{\sc Compiler Construction}\\ TDA282 \\ Submission A}

\author{Andreas Granstr\"om, 870208-8512 \\ \url{andgra@student.chalmers.se} \and Viktor Almqvist, YYMMDD-XXXX \\ \url{almqvisv@student.chalmers.se}}
\begin{document}
\maketitle


\section{About}
For Submission A we have implemented not only the single dimension
array extension, but also the multi-dimensional array (MDA)
extension. We are aware that this will not give extra credit, but it
felt natural to add the functionallity since we added support for MDA
in the typechecker in preperation for Submission B. It was then a
small task to add support for MDA to the code generation.

\section{Usage}
\texttt{jlc <source file>} \\ This will produce a java class file
(extension \texttt{.class}) and a jasmin source file (extension
\texttt{.j}) in the same directory as the supplied source file.

\subsection{Options}
One can pass the \texttt{-t} option to the compliler (ex. \texttt{jlc
  -t <source file>}), this will print (to standard output) the
type-annotated abstract syntaxt-tree that the typechecker pass to the
code-generator. This is mostly used for debugging purpuses. Note, this
will not comiple the source file.

\subsection{Building}
To bulild the compiler, run \texttt{make} in the \texttt{src} directory, it will then produce the \texttt{jlc} binary in the directory ``above''.

The available targets are
\begin{description}
\item[all] Build everything.
\item[jlc] Only build the compiler (not the parser-part).
\item[runtime] Compile the Runtime class.
\item[bnfc] Only build parser-related stuff (Happy and Alex files).
\item[clean] Remove binaries etc.
\item[distclean] Remove all intermediate files, including Happy and Alex files.

\end{description}


\section{The Javalette Language}
After running \texttt{make} in the \texttt{src} dir, view the
\texttt{DocJavalette.txt} file (or build the \texttt{DocJavalette.tex}
file with \LaTeX) for documentation of the language.

\section{Parser issues}
\subsection{shift/reduce}
We have one shift/reduce error in our parser. It is the ``dangling else''
problem, i.e., we have an ambiguity in our grammar. 

Example:
\begin{lstlisting}
  if (A) if (B) ; else;
\end{lstlisting}

As our language is specified, it is ambigious if the \texttt{else}
belongs to \texttt{if (A)} or \texttt{if (B)}.


\end{document}
